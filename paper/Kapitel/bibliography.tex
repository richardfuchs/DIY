% can use a bibliography generated by BibTeX as a .bbl file
% BibTeX documentation can be easily obtained at:
% http://www.ctan.org/tex-archive/biblio/bibtex/contrib/doc/
% The IEEEtran BibTeX style support page is at:
% http://www.michaelshell.org/tex/ieeetran/bibtex/
%\bibliographystyle{IEEEtran}
% argument is your BibTeX string definitions and bibliography database(s)
%\bibliography{IEEEabrv,../bib/paper}
%
% <OR> manually copy in the resultant .bbl file
% set second argument of \begin to the number of references
% (used to reserve space for the reference number labels box)
\begin{thebibliography}{1}

\bibitem{lcdlayers}
ed g2s
\emph{LCD layers.svg}
\hskip 1em plus 0.5em minus 0.4em\relax
\url{https://en.wikipedia.org/wiki/File:LCD_layers.svg}
(am 1.~Februar aberufen)
CC-BY-SA 

\bibitem{aufbau_und_funktion}
D.~Holke
\emph{Aufbau und Funktion einer LCD Zelle}
\hskip 1em plus 0.5em minus 0.4em\relax
\url{https://www.cmb-systeme.de/technikwissen/aufbau-und-funktion-einer-lcd-zelle}

\bibitem{anleitung}
F.~Oestreicher
\emph{Selbstbau einer funktionsfähigen Flüssigkristallanzeige}
\hskip 1em plus 0.5em minus 0.4em\relax
\url{http://fluessigkristalle.com/selbstbau.htm}

\end{thebibliography}