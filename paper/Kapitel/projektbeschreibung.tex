\section{Projektbeschreibung}
\subsection{Motivation}
In diesem Projekt soll eine selbstgebaute Dezimal-Binär-Uhr hergestellt werden. Diese soll mit Batterien und ohne Netzanbindung betrieben werden, da solche Uhren noch nicht existieren. Hinzu kommt, dass solch eine Uhr energiesparend und kabellos ist. Die Anzeige soll aus einem selbstgebauten Flüssigkristalldisplay (Liquid Crystal Display, LCD) bestehen. Angesteuert wird der Display über einen Microcontroller. Zwei Buttons werden verbaut, um die Uhr minutengenau einstellen zu können.

\subsection{Spezifikationen}
Die Uhr sollte folgende Eigenschaften aufweisen:
\begin{itemize}
\item Anzeige der Uhrzeit in einem Dezimal-Binär-Code (binäre Darstellung der Dezimalstellen)
\item Energiearmer Verbauch mit Batterien
\item Ansteuerung durch Microcontroller
\end{itemize}

\subsection{Produktionsschritte}

Bei der Herstellung des Uhr soll wie folgt vorgegangen werden: Zuerst wird das LCD-Display an sich hergestellt. Nach der Fertigstellung der Elektronik wird das Programm geschrieben und das Gehäuse gefertigt. 

\subsection{Vorraussetzungen}
%FIXME welche noch

Dank des FAUFabLab standen uns diverse Maschinen und Werkzeuge zur Verfügung, die uns die Arbeit sehr erleichtert haben. Oft sind diese nicht in jedem privaten Haushalt zu finden, weshalb der Nachbau sich schwierig gestalten kann.
Im Folgenden sind die verwendeten Werkzeuge aufgeführt:

\begin{itemize}
\item Folienplotter
\item Heißluftfön
\item Backofen
\item Chemikaliengefäße (Schälchen und Glasgefäße)
\item Heizfeld
\item Pipetten
\item Sämtliches Zubehör zur Platinenherstellung
\item Lasercutter
\item Multimeter
\item Schere
\item Lötkolben
\item Seitenschneider
\end{itemize}


Folgende Materialien wurden verwendet:

\begin{itemize}
\item Indium-Zinn beschichtete Glasplatten (2 Stck. pro Display)
\item Aceton
\item Klebefolie für Schneideplotter
\item 5\%-ige Salzsäue
%FIXME Menge
\item Siedesteinchen
\item Reinigungskonzentrat (alkalisch, tensidfrei)
%FIXME wie viel + welches
\item Fusselfreie Tücher
\item Abstandshalterbogen \(15\,\mathrm{\mu m}\)
%FIXME Zigerattenfolien dazuschreiben? - in Anleitung dazuschreiben!
\item Zweikomponentenkleber
\item Nematische Weitbereichsmischung / Flüssigkristall
%FIXME wie viel
\item Polarisationsfolie selbstklebend, transparent
\item Polarisationsfolie selbstklebend, mit Reflektor
\item Zubehör zur Platinenherstellung
\item Acryl
\end{itemize}