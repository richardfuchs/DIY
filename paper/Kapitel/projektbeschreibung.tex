\section{Projektbeschreibung}
\subsection{Motivation}
In diesem Projekt soll eine selbstgebaute Dezimal-Binär-Uhr hergestellt werden. Diese soll mit Batterien und nicht mit Strom betrieben werden, da solche Uhren noch nicht existieren. Hinzu kommt, dass solch eine Uhr energiesparend und kabellos ist. Die Anzeige soll aus einem selbstgebauten Flüssigkristalldisplay (Liquid Crystal Display, LCD) bestehen. Angesteuert wird der Display über einen Microcontroller. Zwei Buttons werden verbaut, um die Uhr Minutengenau einstellen zu können.

\subsection{Spezifikationen}
Die Uhr sollte folgende Eigenschaften aufweisen:
\begin{itemize}
\item Anzeige der Uhrzeit in einem Dezimal-Binär-Code (binäre Darstellung der Dezimalstellen)
\item Energiearmer Verbauch mit Batterien
\item Ansteuerung durch Microcontroller
\end{itemize}

\subsection{Produktionsschritte}

Bei der Herstellung des Uhr soll wie folgt vorgegangen werden: Zuerst wird das LCD-Display an sich hergestellt. Nach der Fertigstellung der Elektronik wird das Programm geschrieben und das Gehäuse gefertigt. 

\subsection{Vorraussetzungen}
Für die Herstellung der Uhr werden folgende Werkzeuge Maschinen gebraucht:
\begin{itemize}
\item Folienplotter
\item Backofen
\item Multimeter
\item Stromversorgung \(5\,\mathrm{V}\) AC

\end{itemize}
Folgende Materialien wurden verwendet:
\begin{itemize}
\item ITO beschichtete Glasplatten
\item Aceton
\item 5\%-ige Salzsäue
\item Siedesteinchen
\item Zweikomponentenkleber
\item Nematische Weitbereichsmischung
\item Reinigungskonzentrat TODO welches
\item Polarisationsfolie transparent
\item 2 Polarisationsfolie (selbstklebend) mit Reflektor
\item 1 Bogen Abstandshalter \(15\,\mathrm{\mu m}\)
\end{itemize}