\section{Projektbeschreibung}
\subsection{Motivation}
In diesem Projekt wird eine selbstgebaute Dezimal-Binär-Uhr hergestellt die mit Batterien und ohne Netzanbindung betrieben wird. Die Anzeige soll aus einem selbstgebauten Flüssigkristalldisplay (Liquid Crystal Display, LCD) bestehen. Dadurch wird die Uhr besonders energiesparend und es können individuelle Symbole zur Anzeige der Uhrzeit gewählt werden. Angesteuert wird der Display über einen Mikrocontroller. Zwei Taster dienen zur minutengenau Einstellung der Uhr. Derartige Binäruhren sind nicht käuflich erhältlich und es existieren noch keine Anlneitungen dafür.

\subsection{Spezifikationen}
Die Uhr sollte folgende Eigenschaften aufweisen:
\begin{itemize}
\item Anzeige der Uhrzeit in einem Dezimal-Binär-Code (binäre Darstellung der Dezimalstellen)
\item Betrieb mit herkömmlichen Batterien
\item Geringer Verbrauch und somit langer unabhängiger Betrieb
\item Individuelles Design
\item Ansteuerung durch einen Mikrocontroller
\end{itemize}

\subsection{Produktionsschritte}

Bei der Herstellung des Uhr wird wie folgt vorgegangen: Zuerst wird das LCD-Display an sich hergestellt. Nach der Fertigstellung der Elektronik wird das Programm geschrieben und das Gehäuse gefertigt. 

%\subsection{Vorraussetzungen}

%Dank des \cite[FAUFabLab]{fablabwerkzeuge} standen uns diverse Maschinen und Werkzeuge zur Verfügung, die uns die Arbeit sehr erleichtert haben. Oft sind diese nicht in jedem privaten Haushalt zu finden, weshalb der Nachbau sich schwierig gestalten kann.
%Im Folgenden sind die verwendeten Werkzeuge aufgeführt:\\

%\begin{itemize}
%\item Folienplotter
%\item Heißluftfön
%\item Backofen
%\item Chemikaliengefäße (Schälchen und Glasgefäße)
%\item Heizfeld
%\item Pipetten
%\item Zubehör zur Platinenherstellung (Drucker, Belichter, Bohrer)
%\item Lasercutter (Zuschneiden des Gehäuses)
%\item Multimeter
%\item Schere
%\item Lötkolben
%\item Seitenschneider\\
%\end{itemize}

\subsection{Verwendete Materialien}

Folgende Materialien wurden für 1 Display verwendet (Manche Materialien können mehrfach verwendet werden):\\

\begin{itemize}
\item Indium-Zinn beschichtete Glasplatten (2 Stück pro Display)
\item Aceton
\item Klebefolie für Schneideplotter
\item 5\%-ige Salzsäue (200ml, kann wiederverwendet werden)
\item Siedesteinchen (10-20 Stück)
\item Reinigungskonzentrat RBS 35 (100g, kann wiederverwendet werden)
\item Fusselfreie Tücher
\item Abstandshalterbogen \(15\,\mathrm{\mu m}\) (4 Stück 1mmx4mm)
\item Zweikomponentenkleber
\item Nematischer Flüssigkristall (2g)
\item Polarisationsfolie selbstklebend, transparent
\item Polarisationsfolie selbstklebend, mit Reflektor
\item Zubehör zur Platinenherstellung (Platinen, Chemikalien)
\item Acryl zur Gehäuseherstellung
\end{itemize}